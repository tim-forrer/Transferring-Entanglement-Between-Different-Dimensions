\subsection{Entanglement}
Entanglement is a property of quantum systems that is the source of many of the remarkable results displayed by quantum algorithms.
Essentially it is the presence of correlations in a quantum system that is stronger than possible classically.
Although entanglement can exist between multipartite systems, such as in the case of GHZ states or W states, in this report only bipartite entanglement will be discussed.
Therefore, all references to entanglement are specifically referring to bipartite entanglement.
Generalising entanglement to higher dimensions in the bipartite setting is done by having entanglement between qudits as opposed to qubits.
There is no requirement for the dimensions of the entangled qudits to be matching.
For example,
\begin{equation}
    \ket{\psi} = \frac{1}{\sqrt{2}}\left(\ket{0}_2\otimes\ket{0}_3 + \ket{1}_2\otimes\ket{1}_3\right),
\end{equation}
is a entangled state between a qubit and a qutrit ($d=3$).

\subsubsection{Measuring entanglement}
\label{subsubsection:measure_entanglement}
In order to analyse the efficiency of an entanglement transfer protocol, it is useful to have a method of measuring entanglement, of which numerous methods have been proposed and are used.
For a full mathematical discussion on entanglement quantification, see \cite{Plenio_2007}.
In this report, the measure of entanglement used is the \emph{logarithmic negativity}, first presented by Vidal \cite{Vidal_2002}.
\begin{definition}[Logarithmic Negativity]
    \label{definition:log_neg}
    Let $\rho$ be a density matrix describing a bipartite quantum system $\mathcal{H} = \mathcal{H}_A \otimes \mathcal{H}_B$.
    The logarithmic negativity, $E_\mathcal{N}(\rho)$, of $\rho$ is defined as
    \begin{equation}
        E_\mathcal{N}(\rho) = \log_2\left\Vert \rho^{\Gamma_A}\right\Vert_1.
    \end{equation}
    where $\rho^{\Gamma_A}$ denotes the partial transpose of $\rho$ with respect to the subsystem $\mathcal{H}_A$ and $\Vert X\Vert_1 = Tr(\sqrt{X^\dagger X})$ denotes the trace norm of $X$.
\end{definition}
The logarithmic negativity is also referred to as the log negativity.
It is an appropriate choice for the purpose of this report as it is an entanglement measure that is computable for mixed as well as pure states, and satisfies (strong) additivity
\begin{equation}
    E_\mathcal{N}(\rho \otimes \sigma) = E_\mathcal{N}(\rho) + E_\mathcal{N}(\rho).
\end{equation}
A proof for this is given in \cite{Vidal_2002}.\newline
This usefulness of this can be demonstrated with the use of the following claim.
\begin{claim}
    \label{claim:maximally_entangled_states}
    A maximally entangled state in $d\times d$ dimensions has a log negativity of $\log_2d$.
\end{claim}
\begin{proof}
    First, note that for a Hilbert space of dimension $d\times d$, the state
    \begin{equation}
        \ket{\psi} = \frac{1}{\sqrt{d}}\sum_{j=0}^{j=d-1}\ket{j}\otimes\ket{j}\in \mathcal{H}_A \otimes \mathcal{H}_B
    \end{equation}
    is maximally entangled, since its reduced density matrix in either subspace is maximally mixed.
    So to quantify the entanglement of any maximally entangled state it is sufficient to calculate the log negativity of the density matrix, $\rho$ associated with $\ket{\psi}$.
    Since $\ket{\psi}$ is a pure state
    \begin{align}
        \rho &= \ket{\psi}\bra{\psi}\\
        &=\frac{1}{d} \sum_{j,k}\ket{j}\bra{k}\otimes\ket{j}\bra{k}
    \end{align}
    To calculate $E_{\mathcal{N}}(\rho)$ we need the partial transpose on one of the subspaces, the choice of which is irrelevant since they are both equally entangled with one another.
    So choose to find the partial transpose in $\mathcal{H}_A$.
    \begin{align}
        \rho^{\Gamma_A} =\frac{1}{d} \sum_{j,k}\ket{k}\bra{j}\otimes\ket{j}\bra{k}
    \end{align}
    From this we wish to calculate the trace norm $\Vert X\Vert_1 = \Tr\left(\sqrt{XX^{\dagger}}\right)$.
    \begin{align}
        \rho^{\Gamma_A}\left(\rho^{\Gamma_A}\right)^{\dagger} &= \frac{1}{d^2} \sum_{j,k}\ket{k}\bra{k}\otimes\ket{j}\bra{j}\\
        &=\frac{1}{d^2}I_d\otimes I_d\\
        &=\frac{1}{d^2}I_{d^2}\\
        \implies \Vert\rho^{\Gamma_A}\Vert_1 &= \frac{1}{d}\Tr{\left(I_{d^2}\right)}\\
        &= \frac{1}{d}\cdot d^2\\
        &= d
    \end{align}
    Therefore, this gives
    \begin{align}
        E_{\mathcal{N}} &= \log_2\left\Vert \rho^{\Gamma_A}\right\Vert\\
        &= \log_2d.
    \end{align}
\end{proof}
This combined with the strong additivity condition is what makes the choice of log negativity an apt one.
In comparing how efficient the transfer of entanglement from the Bell pairs to a qudit pair, the log negativity allows for direct comparison of the entanglement transferred into the two qudit system, and the entanglement of the resultant state of the qudit pair.
If the entanglement in $n$ Bell states is transferred, then the final state of the qudit pair has log negativity $n$ in the case of optimal transfer.