\subsection{Entanglement}
Entanglement is a property of quantum systems that is the source of many of the remarkable results displayed by quantum algorithms.
Essentially it is the presence of correlations in a quantum system that is stronger than possible classically.
Although entanglement can exist between multipartite systems, in this report only bipartite entanglement will be discussed.
Therefore, all references to entanglement are specifically referring to bipartite entanglement.
\begin{definition}[Separable Pure States]
    Consider two qubits $A$ and $B$, residing in Hilbert spaces $\mathcal{H}_A$, $\mathcal{H}_B$. Their composite state $\ket{\Psi}$ resides in the composite Hilbert space $\mathcal{H}_A \otimes \mathcal{H}_B$.
\end{definition}
\begin{definition}[Entangled States]
    Two qubits $A$ and $B$ are entangled if their composite state cannot be expressed as a seperable state.
\end{definition}
Generalising entanglement to higher dimensions in the context of this report is done by having entanglement between qudits as opposed to qubits. There is no requirement for the dimensions of the entangled qudits to be matching. For example,
\begin{equation}
    \ket{\psi} = \frac{1}{\sqrt{2}}\left(\ket{0}_2\otimes\ket{+_0}_3 + \ket{1}_2\otimes\ket{+_1}_3\right),
\end{equation}
is a entangled state between a qubit and a qutrit ($d=3$).

\subsubsection{Measuring entanglement}
\label{subsubsection:measure_entanglement}
In order to analyse the efficiency of an entanglement transfer protocol, it is useful to have a method of quantifying entanglement, of which numerous methods have been proposed and are used.
For a full mathematical discussion on entanglement quantification, see \cite{Plenio_2007}.
Here we will introduce two methods of quantifying entanglement.
The first is the \emph{negativity}.
\begin{definition}[Negativity]
    \label{definition:negativity}
    The negativity of a density matrix $\rho$ is given by
    \begin{equation}
        \mathcal{N}(\rho) = \frac{||\rho^{\Gamma_A}||_1-1}{2},
    \end{equation}
    where $\rho^{\Gamma_A}$ denotes the partial transpose of $\rho$ with respect to the subsystem $\mathcal{H}_A$ and $\Vert X\Vert_1 = Tr(\sqrt{X^\dagger X})$ denotes the trace norm of $X$.
\end{definition}
The second method is closely related to the negativity and is called the \emph{logarithmic}, or \emph{log}, \emph{negativity}.

\begin{definition}[Logarithmic Negativity]
    \label{definition:log_neg}
    The logarithmic negativity, $E_\mathcal{N}(\rho)$, of a system described by the density matrix $\rho$ is given by
    \begin{equation}
        E_\mathcal{N}(\rho) = \log_2\left\Vert \rho^{\Gamma_A}\right\Vert_1.
    \end{equation}
\end{definition}
From these definitions the relationship between $\mathcal{N}$ and $E_{\mathcal{N}}$ is easy to derive,
\begin{equation}
    E_\mathcal{N} = \log_2\left(2\mathcal{N} + 1\right).
\end{equation}
One key difference between these two measures is they way in which they quantify entanglement of maximally entangled states.
Bell states, which are maxially entangled two qubit states, have a negativity of $\frac{1}{2}$, and a log negativity of $1$.
Consider now the maximally entangled ququart state ($d = 4$),
\begin{equation}
    \frac{1}{2}\left(\ket{00} + \ket{11} + \ket{22} + \ket{33}\right).
\end{equation}
This state can be considered equivalent to two Bell states, given that the state space is of the same dimension, 16, as two Bell states.
Therefore, it would be desirable for our entanglement quantifier measure the entanglement in the ququart state to be twice the value for the Bell state, in order to make comparisons for entanglement transferred.
Using the negativity, the ququart gives a value of 1.5.
The log negativity however gives a value of 2.
Hence, when quantifying entanglement transferred, in this report we will use the log negativity as it will allow for direct comparison.
As with all quantities that can be measured there is a unit of bipartite entanglement.
\begin{definition}[Ebit]
    \label{definition:ebit}
    The ebit is defined as the entanglement present in a maximally entangled two qubit state (a Bell state).
\end{definition}
Note that this definition is free from specifying a measure of entanglement, so it is not always the case that 2 ebits of entanglement is present in a state equivalent to 2 Bell states (the negativity is an example of this).
As shown previously though, this is the case in the log negativity and indeed, 1 ebit is equal to 1 unit of log negativity.

\begin{claim}
    \label{claim:maximally_entangled_states}
    A maximally entangled state in $d\times d$ dimensions has $\log_2d$ ebits when measured with log negativity.
\end{claim}
\begin{proof}
    First, note that for a Hilbert space of dimension $d\times d$, the state
    \begin{equation}
        \ket{\psi} = \frac{1}{\sqrt{d}}\sum_{j=0}^{j=d-1}\ket{j}\otimes\ket{j}
    \end{equation}
    is maximally entangled, since its reduced density matrix is maximally mixed.
    Therefore, to calculate the entanglement of a maximally mixed state it is sufficient to measure the entanglement of $\ket{\psi}$.
    The density matrix of $\ket{\psi}$ is 
    \begin{align}
        \rho &= \ket{\psi}\bra{\psi}\\
        &=\frac{1}{d} \sum_{j,k}\ket{j}\bra{k}\otimes\ket{j}\bra{k}
    \end{align}
    To calculate $E_{\mathcal{N}}$ we need the partial transpose on one of the subsystems, the choice of which is irrelevant since they are both entangled with one another.
    So we choose to find the partial transpose on $A$.
    \begin{align}
        \rho^{\Gamma_A} =\frac{1}{d} \sum_{j,k}\ket{k}\bra{j}\otimes\ket{j}\bra{k}
    \end{align}
    From this we wish to calculate the trace norm 
    \begin{equation}
        \Vert\rho\Vert_1 = \sqrt{\rho\rho^{\dagger}}.
    \end{equation}
    \begin{align}
        \rho^{\Gamma_A}\left(\rho^{\Gamma_A}\right)^{\dagger} &= \frac{1}{d^2} \sum_{j,k}\ket{k}\bra{k}\otimes\ket{j}\bra{j}\\
        &=\frac{1}{d^2}I_d\otimes I_d\\
        &=\frac{1}{d^2}I_{d^2}\\
        \implies \Vert\rho^{\Gamma_A}\Vert_1 &= \frac{1}{d}\Tr{\left(I_{d^2}\right)}\\
        &= \frac{1}{d}\cdot d^2\\
        &= d
    \end{align}
    Therefore, this gives us that
    \begin{align}
        E_{\mathcal{N}} &= \log_2\left\Vert \rho^{\Gamma_A}\right\Vert\\
        &= \log_2d.
    \end{align}
\end{proof}