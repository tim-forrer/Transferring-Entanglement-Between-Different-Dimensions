\subsection{Ancilla-Based Quantum Computing}
Quantum walks is but one of many universal quantum computing models.
Another model is known as \emph{ancilla-based quantum computing}, which, in particular, aims to resolve two conflicting demands when building quantum computers:
We wish for our qubits to be well isolated to prevent decoherence, yet also want them to interact with each other to perform our quantum computations.
However, a qubit cannot distinguish between unwanted and wanted interactions, resulting in the need for a balancing act that adequetely addresses these two issues. \paragraph{}

Ancilla-based quantum computing aims to resolve this conflict by using additional qubits, known as ancilla qubits, which mediate interactions between the main register qubits. 
Using this model, it is possible to implement two complementary registers of qubits.
The main register can be designed to be strongly isolated from interactions, bar a select few natural interactions with the ancilla registry, whose qubits can be easily manipulated but decohere much faster. Quantum computations can be performed by \emph{delocalising} quantum information from the main register across both the register and the ancilla. After performing a computation on the ancilla qubit, the quantum information is then \emph{relocalised} back into the main register.
The ancilla can be reset to the initial state and used again for other computations.
A simple example of a circuit designed for ancilla-based quantum computing is shown in Figure \ref{simple aqc}. 

\begin{figure}
    \begin{center}
        \begin{tikzpicture}
            \node[scale=1.0] {
                \begin{quantikz}
                    \lstick{$\ket{\psi}$} & \ctrl{1} & \qw &\ctrl{1} & \qw\\
                    \lstick{$\ket{0}$} & \targ{} & \gate{Z} & \targ{} & \qw
                \end{quantikz}
                $\equiv$
                \begin{quantikz}
                    \lstick{$\ket{\psi}$} & \gate{Z} & \qw
                \end{quantikz}
            };
        \end{tikzpicture}
    \caption{Two circuits that implement a $Z$ gate acting on an arbitrary qubit $\ket{\psi}$.}
    \label{simple aqc}
    \end{center}
    \end{figure}
The circuit on the left hand side of Figure \ref{simple aqc} does not directly implement any unitary transformations on an arbitrary qubit state $\ket{\psi} = \alpha \ket{0} + \beta \ket{1}$, but instead indirectly acts on $\ket{\psi}$ via the ancilla qubit, to which the quantum information of $\ket{\psi}$ is shared to via a CNOT gate. Acting a $Z$ gate on the ancilla and relocalising the quantum information with another CNOT gate "passes on" the effects of the $Z$ gate.\paragraph{}

Later in this report I will highlight how this particular model of quantum computing is well suited for the protocol outlined in \cite{giordani2020}, which will be discussed in detail in the following section.
