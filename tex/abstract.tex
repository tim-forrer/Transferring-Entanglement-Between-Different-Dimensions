\begin{abstract}              
 
    Entangled qudit states, which share higher dimensional entanglement, enable quantum computers to unlock greater advantages in the computations that they are able to perform.
    As such, the generation of these entangled states is a worthy problem to consider as quantum computing technology becomes more and more advanced.
    Many methods of generating Bell pair entangled qubits have been proposed and realised, but schemes for qudit pairs of arbitrary dimension are not as well-researched.
    A protocol utilising quantum walk dynamics has been proposed to repeatedly transfer bipartite entanglement from Bell pairs to be accumulated in qudit pairs.
    Analysis of this protocol shows that this transfer is not optimal when using true quantum walk dynamics.
    Due to its non-unitary nature, it is also not easily reversed to retrieve entanglement from higher dimensional entangled states. Post-selection is also needed, where states are sometimes discarded and the entire protocol has to be repeated.
    An adapted version of the protocol, given as a circuit of gates conforming to ancilla-based quantum computing constraints rather than quantum walk ones, is proposed as an alternative which is able to achieve optimal entanglement transfer.
    It allows for entanglement to be transferred freely between qubit and qudit pairs as it is fully unitary and hence reversible.
    Furthermore, it is shown that with the addition of a single gate, the circuit can be used to turn qudits into a form of quantum random access memory.
    
\end{abstract}