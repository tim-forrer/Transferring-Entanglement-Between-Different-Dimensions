Although the goal of this research was to design a protocol in a similar setting to the QW based protocol that could optimally generate maximally entangled states, further probing of the AQC circuit shows that it has further uses beyond storage of entanglement.
In this section, consider a scenario where the quantum information contained in a collection of qubits is to be stored.
By taking one half of the circuit given for the two lab setting and adding a gate $U^{-1}$ that undoes the basis state map seen before, as shown in figure \ref{fig:aqc_qbit_store}, arbitrary qubit states can be stored in the qudit, provided that $d \geq 2^n$, where $n$ is the number of qubit states to be stored.
Furthermore, in the case that two or more qubit states are stored via this circuit then it is in fact possible to retrieve the qubit states in a different order to that in which they were stored, as long as the original order they were stored in is known.
Therefore, this circuit can be utilised in turning qudits into quantum random access memory.

\begin{figure}
    \begin{center}
        \begin{tikzpicture}
            \node[scale=1.0] {
                \begin{quantikz}
                    \lstick{$\ket{d}$} & \gate{F} & \ctrl{1} & \gate{F^{-1}} & \gate{U} & \ctrl{1} & \qw & \gate{U^{-1}} & \qw\\
                    \lstick{$\ket{b}$} & \qw & \gate{\Omega} & \qw & \qw & \gate{X} & \qw & \qw & \qw\\
                \end{quantikz}
            };
        \end{tikzpicture}
    \caption{The AQC circuit with one qudit and qubit which can store arbitrary qubit states in the qudit.}
    \label{fig:aqc_qbit_store}
    \end{center}
\end{figure}

\subsection{Quantum Random Access Memory}
\label{subsection:qram}
The procedure for utilising the circuit as a quantum random access memory is as follows.
Assume that there is a qudit in state $\ket{0}_d$ and two qubits, qubit 1 and qubit 2, in states $a\ket{0}_2 + b\ket{1}_2$ and $c\ket{0}_2 + d\ket{1}_2$ respectively.
We first run the circuit with the qudit and qubit 1,
\begin{equation}\label{eqn:single_qbit_store}
    \ket{0}_d \otimes \left(a\ket{0} + b\ket{1}\right) \longrightarrow \left(a\ket{0}_d + b\ket{1}_d\right) \otimes \ket{0}_2.
\end{equation}
We then replace qubit 1 with qubit 2 and run the circuit again,
\begin{equation}
    \left(a\ket{0}_d + b\ket{1}_d\right) \otimes \left(c\ket{0}_2 + d\ket{1}_2 \right) \longrightarrow \left(ac\ket{0}_d + bc\ket{1}_d + ad\ket{2}_d + bd\ket{3}_d \right)\otimes\ket{0}_2.
    \label{eqn:twoqbitsinqdit}
\end{equation}
In order to retrieve qubit 2 back, this can be done simply by running the circuit backwards.
However, if we wish to retrieve the qubit 1 state then we first require a specific unitary operator. The form of the unitary transform can be found as follows.
Rewrite each of the qudit basis state numbers in binary, i.e.
\begin{align}
    \ket{0} &= \ket{00}\\
    \ket{1} &= \ket{01}\\
    \ket{2} &= \ket{10}\\
    \ket{3} &= \ket{11}.
\end{align}
Rewriting the final state of equation \ref{eqn:twoqbitsinqdit} in this way we obtain the following expression
\begin{equation}
    ac\ket{0} + bc\ket{1} + ad\ket{2} + bd\ket{3} = ac\ket{00} + bc\ket{01} + ad\ket{10} + bd\ket{11},
    \label{eqn:binaryexpression}
\end{equation}
which can be rewritten as the product state
\begin{equation}
    \underbrace{\left(c\ket{0} + d\ket{1}\right)}_{\text{Qubit 2}} \otimes \underbrace{\left(a\ket{0} + b\ket{1}\right)}_{\text{Qubit 1}}.
\end{equation}
Quite remarkably the qudit has an intuitive alternative expression as the product state of qubit 1 and qubit 2.
If the circuit is run backwards, we will retrieve qubit 2.
Therefore, to retrieve qubit 1 instead, the qudit state should be expressable as
\begin{equation}
    \underbrace{\left(a\ket{0} + b\ket{1}\right)}_{\text{Qubit 1}} \otimes \underbrace{\left(c\ket{0} + d\ket{1}\right)}_{\text{Qubit 2}}.
\end{equation}
This can be thought of as switching the positions of our two qubits, which leads us to the unitary transformation we need.
We need a map, $M$, which will switch the positions of our two qubits.
$M$ is found by mapping each of the binary qudit state representations $\{\ket{00}, \ket{01}, \ket{10}, \ket{11}\}$ to the basis state with the binary digits switched.
\begin{align}
    \ket{00} &\mapsto \ket{00}\\
    \ket{01} &\mapsto \ket{10}\\
    \ket{10} &\mapsto \ket{01}\\
    \ket{11} &\mapsto \ket{00}.
\end{align}
We can check that this gives us the form that we want taking the RHS of equation \ref{eqn:binaryexpression} and acting $M$ on it to obtain
\begin{align}
    M\left(ac\ket{00} + ad\ket{01} + bc\ket{10} + bd\ket{11}\right) 
    &=  ac\ket{00} + ad\ket{10} + bc\ket{01} + bd\ket{11}\\
    &= \underbrace{\left(a\ket{0} + b\ket{1}\right)}_{\text{Qubit 1}} \otimes \underbrace{\left(c\ket{0} + d\ket{1}\right)}_{\text{Qubit 2}},
\end{align}
as required.
$M$ can be expressed in terms of the original qudit state labelling by converting the binary back
\begin{align}
    \ket{0} &\mapsto \ket{0}\\
    \ket{1} &\mapsto \ket{2}\\
    \ket{2} &\mapsto \ket{1}\\
    \ket{3} &\mapsto \ket{3}.
\end{align}
The matrix representation of $M$ is given by
\begin{equation}
    M = 
    \begin{pmatrix}
        1 & 0 & 0 & 0\\
        0 & 0 & 1 & 0\\
        0 & 1 & 0 & 0\\
        0 & 0 & 0 & 1\\
    \end{pmatrix}.
\end{equation}
Generalising this to larger numbers of qubits stored is done via the same thought process.
Imagine that there is now third qubit to be stored in the qudit, qubit 3, in the state
\begin{equation}
    e\ket{0} + f\ket{1}.
\end{equation}
When qubits 1, 2 and 3 are stored in that same order, it results in the state
\begin{equation}
    ace\ket{0} + ade\ket{1} + bce\ket{2} + bde\ket{3} + ace\ket{4} + adf\ket{5} + bcf\ket{6} + bdf\ket{7},
\end{equation}
which again, when converted to binary numbers, can be expressed as the product state
\begin{equation}
    \left(e\ket{0} + f\ket{1}\right) \otimes \left(c\ket{0} + d\ket{1}\right) \otimes \left(a\ket{0} + b\ket{1}\right).
\end{equation}
If the state of qubit 1 is needed then $M$ must implement the map
\begin{align}
    \ket{1} = \ket{001} &\mapsto \ket{100} = \ket{4}\\
    \ket{3} = \ket{011} &\mapsto \ket{110} = \ket{6}\\
    \ket{4} = \ket{100} &\mapsto \ket{001} = \ket{1}\\
    \ket{6} = \ket{110} &\mapsto \ket{011} = \ket{3}
\end{align}
with all other basis states mapped to themselves as they are identical when switching the first and last digits of their binary representations.
Having operated $M$ on the qudit state, the circuit can be run backwards in order to retrieve qubit 1. As with retrieving Bell pairs in section \ref{section:aqc_transfer}, any qubit in the state $\ket{0}$ can be used to retrieve qubit 1.

In a practical setting it may be more efficient to build a gate, $P$, that permutes the order of the qubits instead of swapping two digits.
One example of $P$ would be the map that permutes each qubit one place the right,
\begin{align}
    \ket{1} = \ket{001} &\mapsto \ket{100} = \ket{4}\\
    \ket{2} = \ket{010} &\mapsto \ket{001} = \ket{1}\\
    \ket{3} = \ket{011} &\mapsto \ket{101} = \ket{5}\\
    \ket{4} = \ket{100} &\mapsto \ket{010} = \ket{2}\\
    \ket{5} = \ket{101} &\mapsto \ket{110} = \ket{6}\\
    \ket{6} = \ket{110} &\mapsto \ket{011} = \ket{3}.
\end{align}
Again $\ket{0}, \ket{7}$ are mapped to themselves as the permutation does not affect them.
In this way, any of the qubit states can be retrieved by continually applying $P$ until the qubit state to be retrieved is in the right "place".
Utilising a single gate would also be more appropriate for the ancilla-based quantum computing setting where a minimal gate set is desired for the control qudits.
However, this would require multiple applications of the same gate which, without adequate error correction, would lead to greater decoherence of the quantum information than if just a single switching gate $M$ were applied.