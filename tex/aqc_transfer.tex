I will now outline an alternative entanglement transfer scheme designed to operate in the same setting at the QW based protocol, but instead is able to transfer entanglement deterministically and also retrieve it perfectly as well.

A circuit schematic is given in Figure \ref{fig:aqc_circuit_schematic}, and essentially is two copies of the same circuit.

\begin{figure}[h]
    \begin{center}
        \begin{tikzpicture}
            \node[scale=1.0] {
                \begin{quantikz}
                    \lstick{$\ket{d}_A$} & \gate{F} & \ctrl{1} & \gate{F^{-1}} & \gate{U} & \ctrl{1} & \qw\\
                    \lstick{$\ket{b}_A$} & \qw & \gate{\Omega^{j}} & \gate{H} & \qw & \gate{Z} & \gate{H}\\
                    - & - & - & - & - & - & - \\
                    \lstick{$\ket{b}_B$} & \qw & \gate{\Omega^{j}} & \gate{H} & \qw & \gate{Z} & \gate{H}\\
                    \lstick{$\ket{d}_B$} & \gate{F} & \ctrl{-1} & \gate{F^{-1}} & \gate{U} & \ctrl{-1} & \qw\\
                \end{quantikz}
            };
        \end{tikzpicture}
    \caption{The AQC circuit for entanglement transfer.}
    \label{fig:aqc_circuit_schematic}
    \end{center}
\end{figure}
Since the two halves of the circuit are completely seperate from each other, this circuit is suitable for the same experimental setup described for the QW protocol, where it is imagined that there are two labs which are spatially seperated, and have a shared source of Bell state entangled qubits which are sent individually to each lab.
The $F$ gate is one that acts the Quantum Fourier Transform on the qudit basis states,
\begin{equation}
    F\ket{x} = \frac{1}{2^{\frac{d}{2}}} \sum_{y=0}^{d-1} \omega^{xy}\ket{y},
\end{equation}
where $\omega = e^{i\frac{2\pi}{d}}$. The matrix representation of $F$ in the computational basis is given by by the $d\times d$ matrix
\begin{equation}
    F = \begin{pmatrix}
        1 & 1 & 1 & \hdots & 1 & 1\\
        1 & \omega & \omega^2 & \hdots & \omega^{d-2} & \omega^{d-1}\\
        1 & \omega^2 & \omega^4 & \hdots & \omega^{2(d-2)} & \omega^{2(d-1)}\\
        \vdots & \vdots & \vdots & & \vdots & \vdots\\
        1 & \omega^{d-1} & \omega^{2(d-1)} & \hdots & \omega^{(d-2)(d-1)} & \omega^{(d-1)^2}
    \end{pmatrix}
\end{equation}

The gate $\Omega$ is given by
\begin{equation}
    \Omega = 
    \begin{pmatrix}
        1 & 0\\
        0 & \omega
    \end{pmatrix},
\end{equation}
therefore the controlled-$\Omega$ gates are ones that act as 
\begin{align}
    C\text{-}\Omega^j (\ket{x}_d\ket{y}_2) &= \ket{x}_d\otimes \Omega^{xj}\ket{y}_2\\
    & = \ket{x}_d\otimes \omega^{xyj}\ket{y}_2
\end{align}
where $\ket{x}_d$ is the control qudit and $\ket{y}_2$ the target qubit.

\begin{proposition}
    \label{prop:COmegaCXinQFTbasis}
    In the Fourier basis, the $C\text{-}\Omega^j$ operator acts on the product state $\ket{+_k}_d\otimes\ket{1}_2$ to give $\ket{+_{k+j}}_d\otimes\ket{1}_2$
\end{proposition}
\begin{proof}
    The Fourier basis state $\ket{+_k}_d$ is given by
    \begin{equation}
        \ket{+_k}_d = \frac{1}{\sqrt{d}}\sum_{m=0}^{d-1}\omega^{km}\ket{m}.
    \end{equation}
    Therefore
    \begin{align}
        C\text{-}\Omega^j\left(\ket{+_k}_d\ket{1}_2\right) 
        &= \frac{1}{\sqrt{d}}\sum_{m=0}^{d-1}C\text{-}\Omega^j\left(\omega^{km}\ket{m}_d\otimes\ket{1}_2\right)\\
        &= \frac{1}{\sqrt{d}}\sum_{m=0}^{d-1}\omega^{mj}\left(\omega^{km}\ket{m}_d\otimes\ket{1}_2\right)\\
        &= \frac{1}{\sqrt{d}}\sum_{m=0}^{d-1}\omega^{(k+j)m}\ket{m}_d\otimes\ket{1}_2\\
        &= \left(\frac{1}{\sqrt{d}}\sum_{m=0}^{d-1}\omega^{(k+j)m}\ket{m}_d\right)\otimes\ket{1}_2\\
        &= \ket{+_{k+j}}_d\otimes\ket{1}_2.
    \end{align}
\end{proof}

Proposition \ref{prop:COmegaCXinQFTbasis} highlights the somewhat counterintuitive nature of the circuit, and indeed the counterintuitive way ancilla-based quantum computing can exploit quantum phenomena.
We act with a controlled operation, where the qudit is the control and the qubit is the target, yet it is the qudit state that is changed and the qubit is left "untouched".
This idea is crucial for this proposal to achieve the desired outcomes, as we are required to shift the qudit states, but are unable to do so without being in the Fourier basis, where delocalised phase changes shift the Fourier basis states.
Analogously to the QW protocol, the qudit can be thought of as a walker driven by the ancilla qubit "coin".

\subsection{Transfer}
\label{subsection:aqcstorage}
Assume that we have two qudits in labs A and B which are spatially separated, and each lab shares a qubit each from an entanglement Bell pair. The composite state is therefore given by
\begin{equation}
    \ket{0}^{(A)}_d\ket{0}^{(B)}_d\otimes\frac{1}{\sqrt{2}}\left(\ket{00}_2^{(A,B)} + \ket{11}_2^{(A,B)}\right).
\end{equation}
The qudits are Fourier transformed into the Fourier conjugate basis to give
\begin{align}
    &\ket{+_0}^{(A)}_d\ket{+_0}^{(B)}_d\otimes\frac{1}{\sqrt{2}}\left(\ket{00}_2^{(A,B)} + \ket{11}_2^{(A,B)}\right)\\
    =&\frac{1}{\sqrt{2}}\left(
        \ket{+_0}^{(A)}_d\ket{+_0}^{(B)}_d \otimes \ket{00}_2^{(A,B)}
        +\ket{+_0}^{(A)}_d\ket{+_0}^{(B)}_d \otimes \ket{11}_2^{(A,B)}
        \right).
\end{align}
Using Proposition \ref{prop:COmegaCXinQFTbasis}, a single application of the controlled-$\Omega$ in both $A$ and $B$ gives the state
\begin{equation}
    \frac{1}{\sqrt{2}}\left(
        \ket{+_0}^{(A)}_d\ket{+_0}^{(B)}_d \otimes \ket{00}_2^{(A,B)}
        +\ket{+_1}^{(A)}_d\ket{+_1}^{(B)}_d \otimes \ket{11}_2^{(A,B)}
        \right).
\end{equation}
The qudits are then transformed back into the computational basis and the qubits transformed into the Hadamard basis giving,
\begin{equation}
    \frac{1}{\sqrt{2}}\left(
        \ket{0}^{(A)}_d\ket{0}^{(B)}_d \otimes \ket{++}_2^{(A,B)}
        +\ket{1}^{(A)}_d\ket{1}^{(B)}_d \otimes \ket{--}_2^{(A,B)}
        \right).
\end{equation}
In this instance, the map $U$ can be ignored since all the even state qudits are paired with $\ket{+}$ and odd state qudits paired with $\ket{-}$. Applying the controlled Z gate therefore gives us
\begin{equation}
    \frac{1}{\sqrt{2}}\left(
        \ket{0}^{(A)}_d\ket{0}^{(B)}_d \otimes \ket{++}_2^{(A,B)}
        +\ket{1}^{(A)}_d\ket{1}^{(B)}_d \otimes \ket{++}_2^{(A,B)}
        \right).
\end{equation}
Applying the final Hadamard gates on the qubits therefore gives us
\begin{equation}
    \label{eq:singletransferfinalstate}
    \frac{1}{\sqrt{2}}\left(
        \ket{0}^{(A)}_d\ket{0}^{(B)}_d +\ket{1}^{(A)}_d\ket{1}^{(B)}_d\right) \otimes \ket{00}_2^{(A,B)}.
\end{equation}

\subsection{Accumulation}
\label{subsection:accumulation}
Now it is assumed that one ebit of entanglement has been transferred to our qudit and we wish to transfer a further Bell pair into the qudits. Taking the final state given by 
\subsection{Retrieval}
\label{subsection:aqcretrieval}
Given that this is a circuit that solely utilises unitary transformations, retrieval of entangled Bell pairs is trivially done by running the circuit backwards. Furthermore, there is no requirement to use the same ancilla qubits to retrieve the entanglement. By the end of the circuit the ancilla qubits are in the $\ket{0}$ state, therefore any pair of ancilla qubits in the $\ket{0}$ state may be used to retrieve the entanglement.

\subsection{Further uses of the circuit}
\label{subsection:furtheruses}
Although the goal of this research was to design a protocol in a similar setting to the QW based protocol that outperformed the QW protocol, further probing of the proposed circuit showed that it had further uses beyond storage of entanglement.
In this section, we consider a setting in a single lab that has many qubits and a qudit, and we wishes to store the quantum state of the qubits.
By taking one half of the circuit given for the two lab setting and adding a gate $U^{-1}$ that undoes the basis state map seen before, as shown in Figure \ref{fig:aqc_qbit_store}, we can in fact store arbitrary qubit states in our qudits.
Furthermore, in the case that we store two or more qubit states via this circuit then it is in fact possible to retrieve the qubit states in a different order to that which they were stored, simply by keep a record of order in which they were stored.
Therefore, this circuit can be utilised in turning qudits into quantum random access memory.

\begin{figure}
    \begin{center}
        \begin{tikzpicture}
            \node[scale=1.0] {
                \begin{quantikz}
                    \lstick{$\ket{d}$} & \gate{F} & \ctrl{1} & \gate{F^{-1}} & \gate{U} & \ctrl{1} & \qw & \gate{U^{-1}} & \qw\\
                    \lstick{$\ket{b}$} & \qw & \gate{\Omega} & \gate{H} & \qw & \gate{Z} & \gate{H} & \qw & \qw\\
                \end{quantikz}
            };
        \end{tikzpicture}
    \caption{The AQC circuit with one qudit and qubit which can store arbitrary qubit states in the qudit.}
    \label{fig:aqc_qbit_store}
    \end{center}
\end{figure}

\subsubsection{Quantum Random Access Memory}
\label{subsubsection:qram}
The procedure for utilising the circuit as a quantum random access memory is as follows.
Assume that there is a qudit in state $\ket{0}_d$ and two qubits, qubit 1 and qubit 2, in states $a\ket{0}_2 + b\ket{1}_2$ and $c\ket{0}_2 + d\ket{1}_2$ respectively.
We first run the circuit with the qudit and qubit 1,
\begin{equation}
    \ket{0}_d \otimes \left(a\ket{0} + b\ket{1}\right) \longrightarrow \left(a\ket{0}_d + b\ket{1}_d\right) \otimes \ket{0}_2.
\end{equation}
We then replace qubit 1 with qubit 2 and run the circuit again,
\begin{equation}
    \left(a\ket{0}_d + b\ket{1}_d\right) \otimes \left(c\ket{0}_2 + d\ket{1}_2 \right) \longrightarrow \left(ac\ket{0}_d + bc\ket{1}_d + ad\ket{2}_d + bd\ket{3}_d \right)\otimes\ket{0}_2.
    \label{eqn:twoqbitsinqdit}
\end{equation}
In order to retrieve qubit 2 back, this can be done simply by running the circuit backwards.
However, if we wish to retrieve the qubit 1 state then we first require a specific unitary operator. The form of the unitary transform can be found as follows.
Rewrite each of the qudit basis state numbers in binary, i.e.
\begin{align}
    \ket{0} &= \ket{00}\\
    \ket{1} &= \ket{01}\\
    \ket{2} &= \ket{10}\\
    \ket{3} &= \ket{11}.
\end{align}
Rewriting the final state of equation \ref{eqn:twoqbitsinqdit} in this way we obtain the following expression
\begin{equation}
    ac\ket{0} + bc\ket{1} + ad\ket{2} + bd\ket{3} = ac\ket{00} + bc\ket{01} + ad\ket{10} + bd\ket{11},
    \label{eqn:binaryexpression}
\end{equation}
which can be rewritten as the product state
\begin{equation}
    \underbrace{\left(c\ket{0} + d\ket{1}\right)}_{\text{Qubit 2}} \otimes \underbrace{\left(a\ket{0} + b\ket{1}\right)}_{\text{Qubit 1}}.
\end{equation}
Quite remarkably the qudit has an intuitive alternative expression as the product state of qubit 1 and qubit 2.
If the circuit is run backwards, we will retrieve qubit 2.
Therefore, to retrieve qubit 1 instead, the qudit state should be expressable as
\begin{equation}
    \underbrace{\left(a\ket{0} + b\ket{1}\right)}_{\text{Qubit 1}} \otimes \underbrace{\left(c\ket{0} + d\ket{1}\right)}_{\text{Qubit 2}}.
\end{equation}
This can be thought of as switching the positions of our two qubits, which leads us to the unitary transformation we need.
We need a map, $M$, which will switch the positions of our two qubits.
$M$ is found by mapping each of the binary qudit state representations $\{\ket{00}, \ket{01}, \ket{10}, \ket{11}\}$ to the basis state with the binary digits switched.
\begin{align}
    \ket{00} &\mapsto \ket{00}\\
    \ket{01} &\mapsto \ket{10}\\
    \ket{10} &\mapsto \ket{01}\\
    \ket{11} &\mapsto \ket{00}.
\end{align}
We can check that this gives us the form that we want taking the RHS of equation \ref{eqn:binaryexpression} and acting $M$ on it to obtain
\begin{align}
    M\left(ac\ket{00} + ad\ket{01} + bc\ket{10} + bd\ket{11}\right) 
    &=  ac\ket{00} + ad\ket{10} + bc\ket{01} + bd\ket{11}\\
    &= \underbrace{\left(a\ket{0} + b\ket{1}\right)}_{\text{Qubit 1}} \otimes \underbrace{\left(c\ket{0} + d\ket{1}\right)}_{\text{Qubit 2}},
\end{align}
as required.
$M$ can be expressed in terms of the original qudit state labelling by converting the binary back
\begin{align}
    \ket{0} &\mapsto \ket{0}\\
    \ket{1} &\mapsto \ket{2}\\
    \ket{2} &\mapsto \ket{1}\\
    \ket{3} &\mapsto \ket{3}.
\end{align}
The matrix representation of $M$ is given by
\begin{equation}
    M = 
    \begin{pmatrix}
        1 & 0 & 0 & 0\\
        0 & 0 & 1 & 0\\
        0 & 1 & 0 & 0\\
        0 & 0 & 0 & 1\\
    \end{pmatrix}.
\end{equation}
Generalising this to larger numbers of qubits stored is done via the same thought process. Taking the example of storing three qubits,
\begin{align*}
    \text{Qubit 1 } &a\ket{0} + b\ket{1}\\
    \text{Qubit 2 } &c\ket{0} + d\ket{1}\\
    \text{Qubit 3 } &e\ket{0} + f\ket{1}.
\end{align*}
When stored in the order above, the qudit state is given by
\begin{equation}
    ace\ket{0} + ade\ket{1} + bce\ket{2} + bde\ket{3} + ace\ket{4} + adf\ket{5} + bcf\ket{6} + bdf\ket{7},
\end{equation}
which again, when converted to binary numbers, can be expressed as the product state
\begin{equation}
    \left(e\ket{0} + f\ket{1}\right) \otimes \left(c\ket{0} + d\ket{1}\right) \otimes \left(a\ket{0} + b\ket{1}\right).
\end{equation}
If we want to retrive qubit 1 then we require $M$ to perform the following map
\begin{align}
    \ket{1} = \ket{001} &\mapsto \ket{100} = \ket{4}\\
    \ket{3} = \ket{011} &\mapsto \ket{110} = \ket{6}\\
    \ket{4} = \ket{100} &\mapsto \ket{001} = \ket{1}\\
    \ket{6} = \ket{110} &\mapsto \ket{011} = \ket{3}
\end{align}
with all other basis states mapped to themselves as they are identical when switching the first and last digits of their binary representations.
Having operated $M$ on the qudit state, we can now run the circuit backwards in order to retrieve qubit 1. As with retrieving Bell pairs, any qubit in the state $\ket{0}$ can be used to retrieve qubit 1.

In a practical setting it may be more efficient to build a gate, $P$, that permutes the order of the qubits instead of swapping two digits.
One example of $P$ would be the map that permutes each qubit one place the right,
\begin{align}
    \ket{1} = \ket{001} &\mapsto \ket{100} = \ket{4}\\
    \ket{2} = \ket{010} &\mapsto \ket{001} = \ket{1}\\
    \ket{3} = \ket{011} &\mapsto \ket{101} = \ket{5}\\
    \ket{4} = \ket{100} &\mapsto \ket{010} = \ket{2}\\
    \ket{5} = \ket{101} &\mapsto \ket{110} = \ket{6}\\
    \ket{6} = \ket{110} &\mapsto \ket{011} = \ket{3}.
\end{align}
Again $\ket{0}, \ket{7}$ are mapped to themselves as the permutation does not affect them.
In this way we would be able to retrieve any of the qubits by continually applying $P$ until the qubit we wish to retrieve is in the right "place".
Utilising a single gate would also be more appropriate for the ancilla-based quantum computing setting where we wish to have a minimal gate set applied to the control qudits.
However, this would require multiple applications of the same gate which, without adequate error correction, would lead to greater decoherence of the quantum information than if just a single switching gate $M$ were applied.