In this section an alternative entanglement transfer scheme is outlined.
It is designed to operate in the same setting at the QW based protocol, where there are two labs, again called $A$ and $B$, which share a common source of entangled qubits and each have their own qudits.
Analogously to the QW protocol, the qudit can be thought of as a walker driven by the ancilla qubit "coin".
The ideas contained in the QW fit well with an AQC model because they emulate the design philosophy behind AQC near perfectly.
Entangled coins have their entanglement transferred and are then replaced - these can be viewed as the ancilla qubits.
The entanglement is transferred and accumulated in the walkers - these can be viewed as long-lived qudits.

The AQC protocol can be succinctly summarised by the circuit schematic given in figure \ref{fig:aqc_circuit_schematic}, and essentially is two copies of the same sub-circuit.

\begin{figure}[h]
    \begin{center}
        \begin{tikzpicture}
            \node[scale=1.0] {
                \begin{quantikz}
                    \lstick{$\ket{0}_d^{(A)}$} & \gate{F} & \ctrl{1} & \gate{F^{-1}} & \gate{U} & \ctrl{1} & \qw\\
                    \lstick{$\ket{BS}_2^{(A)}$} & \qw & \gate{\Omega^{j}} & \qw & \qw & \gate{X} & \qw\\
                    - & - & - & - & - & - & - \\
                    \lstick{$\ket{BS}_2^{(B)}$} & \qw & \gate{\Omega^{j}} & \qw & \qw & \gate{X} & \qw\\
                    \lstick{$\ket{0}_d^{(B)}$} & \gate{F} & \ctrl{-1} & \gate{F^{-1}} & \gate{U} & \ctrl{-1} & \qw\\
                \end{quantikz}
            };
        \end{tikzpicture}
    \caption{The AQC circuit for entanglement transfer. The schematic shows two separated circuits that exist in spatially separated labs, $A$ and $B$. The qudits and qubits belonging to each individual lab are labelled by their superscripts. The qubits form an entangled Bell state (denoted by $BS$).}
    \label{fig:aqc_circuit_schematic}
    \end{center}
\end{figure}
The majority of the gates shown in the schematic are discussed in section \ref{subsubsection:circuits_and_gates}, except for the gate $\Omega^j$.
The action of $\Omega$ in the computational basis is given by the matrix
\begin{align}
    \Omega &= 
    \begin{pmatrix}
        1 & 0\\
        0 & \omega
    \end{pmatrix}\\
    \implies \Omega^j &=
    \begin{pmatrix}
        1 & 0\\
        0 & \omega^j
    \end{pmatrix}.
\end{align}
$\Omega$ is similar to the $Z$ operator in that it applies relative phase difference, but instead of a phase of -1, a phase corresponding to the $d^{\text{th}}$ root of unity is introduced on the $\ket{1}_2$ states, with $d$ corresponding to the dimension of the control qudits.
(Indeed, when $d=2$, $\Omega = Z$.)
Therefore the $\C{\Omega^j}$ operator is one that acts as
\begin{align}
    C\text{-}\Omega^j (\ket{x}_d\ket{y}_2) &= \ket{x}_d\otimes \Omega^{xj}\ket{y}_2\\
    & = \ket{x}_d\otimes \omega^{xyj}\ket{y}_2,
\end{align}
where $\ket{x}_d$ is the control qudit and $\ket{y}_2$ the target qubit.

\begin{claim}
    \label{claim:COmegaCXinQFTbasis}
    The $C\text{-}\Omega^j$ operator acts on the product state $\ket{+_k}_d\otimes\ket{1}_2$ to give $\ket{+_{k+j}}_d\otimes\ket{1}_2$
\end{claim}
A proof for this is given in appendix \ref{appendix:c_omega}.
This exhibits again the counterintuitive nature of AQC discussed in section \ref{subsection:aqc}.
Despite the qudit being the control and the qubit being the target, the only change effected is on the qudit, shifting the Fourier basis states.
Claim \ref{claim:COmegaCXinQFTbasis} also demonstrates the role of the index $j$, which essentially dictates by how many states each Fourier basis state is shifted.

The following sections demonstrate explicitly how the circuit operates.

\subsection{Transfer}
\label{subsection:aqctransfer}
Assume that there are two qudits in the state $\ket{0}_d$ in labs A and B, and each lab shares a qubit each from an entangled Bell pair. The composite state is therefore given by
\begin{equation}
    \ket{00}_d^{(A,B)}\otimes\frac{1}{\sqrt{2}}\left(\ket{00}^{(A,B)}_2 + \ket{11}^{(A,B)}_2\right).
\end{equation}
Note that in the above expression, the qudits and qubits in both labs, $A$ and $B$, are in fact in the same states.
This is true for the entire running of the circuit and as such there is no need to explicitly differentiate the qudits or ancilla qubits in either lab so the superscript labelling is dropped for the rest of this section.

The qudits are first Fourier transformed into the Fourier basis to give
\begin{align}
    &\ket{+_0 +_0}_d\otimes\frac{1}{\sqrt{2}}\left(\ket{00}_2 + \ket{11}_2\right)\\
    =&\frac{1}{\sqrt{2}}\Big(
        \ket{+_0 +_0}_d \otimes \ket{00}_2
        +\ket{+_0 +_0}_d \otimes \ket{11}_2
        \Big).
\end{align}
The change of basis is needed because to have entanglement in the qudits, they must be in superposition of at least two states of non-zero amplitude.
Since only amplitudes can be delocalised across the qudits and ancilla qubits, in order to give other qudit states a non-zero amplitude, it is necessary to be in a basis where the basis states are related to one another by differences in their amplitudes.
The Fourier basis is one such basis where the basis states are in equal superposition of all the computational basis states but differ by the relative phases in the superposition.
Hence the need to be in the Fourier basis when using the delocalised amplitudes to effect changes in the qudit basis states.
The index $j$ on the $\Omega$ has an analogy to the QW protocol where the iteration number dictates how many steps need to be taken in the QW.
Using the same line of reasoning as outlined in claim \ref{claim:min_steps}, on the $n^{\text{th}}$ iteration of the protocol, $j = 2^{n-1}$ so in this instance (where $n=1$) $j=1$.
Using claim \ref{claim:COmegaCXinQFTbasis}, acting $\C{\Omega}$ in both $A$ and $B$ gives the state
\begin{equation}
    \frac{1}{\sqrt{2}}\left(
        \ket{+_0 +_0}_d \otimes \ket{00}_2
        +\ket{+_1 +_1}_d \otimes \ket{11}_2
        \right).
\end{equation}
The qudits are then transformed back into the computational basis,
\begin{equation}
    \frac{1}{\sqrt{2}}\left(
        \ket{0 0}_d \otimes \ket{00}_2
        +\ket{1 1}_d \otimes \ket{11}_2
        \right).
\end{equation}
The operator $U$ is a correctional gate that essentially pairs all the even numbered computational qudit basis states with $\ket{0}_2$ and the odd numbered computational basis states with $\ket{1}_2$.
(A more complete description is given in section \ref{subsection:accumulation}.)
In this instance, $U$ can be ignored since no corrections are needed, so $\C{X}$ is directly applied, giving
\begin{align}
    &\frac{1}{\sqrt{2}}\left(
        \ket{00}_d \otimes \ket{00}_2
        +\ket{11}_d \otimes \ket{00}_2
        \right)\\
    =&\underbrace{
        \frac{1}{\sqrt{2}}\left(\ket{00}_d +\ket{11}_d\right)
    }_{\text{Bell state}}
    \otimes \ket{00}_2.    
    \label{eqn:single_aqc_transfer_final}
\end{align}
As shown in equation \ref{eqn:single_aqc_transfer_final}, the qudits now form a Bell state and the qubits are no longer entangled - the entanglement has been transferred to the qudits.

\subsection{Accumulation}
\label{subsection:accumulation}
Assume that one Bell pair has been transferred to our qudits and a second Bell pair is to be transferred.
Taking the final state given by equation \ref{eqn:single_aqc_transfer_final} and replacing the ancilla qubits with another entangled Bell pair gives
\begin{equation}
    \frac{1}{2}\left(
            \ket{00}_d +\ket{11}_d
        \right)
        \otimes 
        \left(
            \ket{00}_2 +\ket{11}_2
        \right).
\end{equation}
Again, the qudits are transformed to the Fourier basis before applying the $\C{\Omega^j}$ gate.
As noted in the discussion on $\C{\Omega^j}$ above, for this iteration (where $n=2$) $j=2$, giving the state
\begin{equation}
    \frac{1}{2}\Big[
            (\ket{+_0 +_0}_d + \ket{+_1 +_1}_d)\otimes\ket{00}_2
            +(\ket{+_2 +_2}_d + \ket{+_3 +_3}_d)\otimes\ket{11}_2
        \Big].
\end{equation}
After transforming back to the computational basis, the total state becomes
\begin{equation}
    \frac{1}{2}\Big[
            (\ket{00}_d + \ket{11}_d)\otimes\ket{00}_2
            +(\ket{22}_d + \ket{33}_d)\otimes\ket{11}_2
        \Big].
\end{equation}
Unlike the previous example for $n=1$, here the correctional operator $U$ is needed so that the $\C{X}$ will ensure all the qubit states are $\ket{0}_2$, allowing our qubit state to be completely separable from its associated qudit.
In order for the $\C{X}$ to do this, it must act as the identity on $\ket{0}_2$, and implement an $X$ operation on $\ket{1}_2$.
Since $X$ is self-inverse, the control must be in an even labelled basis state (turning $X$ into the identity) when the target is $\ket{0}_2$ and in an odd labelled basis state (leaving $X$ unchanged) when the target is $\ket{1}_2$.
Hence in this case, $U$ must implement
\begin{align}
    \ket{1} &\mapsto \ket{2}\\
    \ket{2} &\mapsto \ket{1},
\end{align}
which then gives the state
\begin{equation}
    \frac{1}{2}\Big[
            (\ket{00}_d + \ket{22}_d)\otimes\ket{00}_2
            +(\ket{11}_d + \ket{33}_d)\otimes\ket{11}_2
        \Big].
\end{equation}
As desired, $U$ has paired even numbered qudit basis states with $\ket{00}_2$ and odd numbered qudit basis states with $\ket{11}_2$.
Therefore after the $\C{X}$ operation the state becomes
\begin{align}
    &\frac{1}{2}
    (\ket{00}_d + \ket{22}_d + \ket{11}_d + \ket{33}_d)
    \otimes\ket{00}_2.
\end{align}
The two qudits are now in an entangled state with a log negativity of 2 (claim \ref{claim:maximally_entangled_states}).
This protocol can be repeated again and again with the sole change needed being the index $j$ of $\C{\Omega^j}$ and the operator $U$.
For $n>2$ there is no unique choice of $U$, as long as the end result is that even numbered states are paired with $\ket{0}_2$ and odd numbered states are paired with $\ket{1}_2$.
This can be re-expressed as requiring odd numbered states less than $\frac{d}{2}$ and even numbered states greater than or equal to $\frac{d}{2}$ to be swapped in some way.

\subsection{Retrieval}
\label{subsection:aqcretrieval}
Given that this is a circuit that solely utilises unitary transformations, retrieval of entangled Bell pairs is trivially done by running the circuit backwards.
Furthermore, there is no requirement to use the same ancilla qubits to retrieve the entanglement.
By the end of the circuit the ancilla qubits are in the $\ket{0}$ state, therefore any pair of ancilla qubits in the $\ket{0}$ state may be used to retrieve the entanglement.

This ease of transfer means that entanglement can actually be transferred between qudit pairs of any dimension.
The entanglement from the source qudit pair can first be transferred to a collection of intermediary ancilla qubits, and then transferred from the ancilla qubits to the target qudit pair.