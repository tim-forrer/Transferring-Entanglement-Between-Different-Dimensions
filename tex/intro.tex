Quantum walks are powerful tools in the landscape of quantum computing. 
Much like their classical analogues, they exhibit many properties that are desirable for computations and are an extremely useful building block for many algorithms designed for quantum computers \cite{shenvi2003}. 
However they also have some very significant divergences from classical random walks, including a quadratic speed up in their spreading, as shown by Fig 1. 
There is also evidence of robust performance even when the quantum computer is not perfectly isolated from its environment, leading to situations where decoherences due to interactions with the environment is beneficial for a given computation. 
It is possible to divide quantum walks into two categories, discrete time and continuous time. 
The latter have been shown to solve a wide range of problems in a number of different settings. 
However, the focus of this report will be on entangled state generation which neccessitates the need for more than one Hilbert subspace in our system, a setting which lends itself much more readily to discrete time quantum walks.\newline

Quantum walks have been shown to have great versitility in the generation and transfer of entanglement, quantum correlations which have no classical analogue, within a quantum system. 
Entanglement is a key resource for many quantum computing protocols \cite{qkd}\cite{qteleport}[Refs teleportation, superdense coding], and is the source of the quadratic speedup evident in so many quantum algorithms that best their classical counterparts.